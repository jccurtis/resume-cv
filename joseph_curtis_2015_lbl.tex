\documentclass[print]{resume} % Add 'print' for b/w

\begin{document}

\header
{Joseph }
{Curtis}
{Nuclear Engineer}
{Radiation Detection \space \faPlusSign \space\space Data Analysis \space \faPlusSign \space\space Modeling \& Simulation}
{1526 Arch St Apt 4 Berkeley, CA 94708, USA\space\space\faHome}
{\href{mailto:jccurtis@lbl.gov}{jccurtis@lbl.gov}\space\space\faEnvelopeAlt}
{(831) 277 6231\space\space\faPhone}
{\href{http://jccurtis.github.io}{jccurtis.github.io}\space\space\faGithub}
{\href{http://www.linkedin.com/in/josephccurtis}{josephccurtis}\space\space\faLinkedinSign}

\section{education}
\begin{entrylist}
	\entry
		{2012--2014 (Fall)}
		{Master of Science {\normalfont Nuclear Engineering (GPA 3.84)}}
		{University of California, Berkeley}
		{
		Thesis: ``Benchmarking the Gamma Ray Sensors on RadMAP''
		\begin{itemize}
			\item Outstanding Graduate Student Instructor Award
			\item Alpha Nu Sigma Membership (Nuclear Engineering Honor Society)
		\end{itemize}
		}
	\entry
		% the Berkeley Radiological Air and Water Monitoring Project
		{2007--2011}
		{Bachelor of Science {\normalfont Nuclear Engineering (GPA 3.86)}}
		{University of California, Berkeley}
		{
		Honors Thesis: ``Measurements and Analysis of Fukushima Fallout by BRAWM'' 
		\begin{itemize}
			\item Department Citation (Top GPA in graduating class)
			\item Tau Beta Pi Membership (Engineering Honor Society)
		\end{itemize}
		}
\end{entrylist}

%\section{awards}
%\begin{entrylist}
%	\entrymin
%		{2014}
%		{Member of Alpha Nu Sigma}
%		{American Nuclear Society Honor Society}
%	\entrymin
%		% Awarded to one graduate student instructor as elected by department faculty and approved by UC Berkeley.
%		{2013}
%		{Outstanding Graduate Student Instructor}
%		{Graduate Division, UC Berkeley}
%		%\entrymin
%		%{2011}
%		%{Graduated with High Honors}
%		%{College of Engineering, UC Berkeley}
%	\entrymin
%		{2011}
%		{Departmental Citation (Top GPA in graduating class)}
%		{Dept of Nuclear Engineering, UC Berkeley}
%	\entrymin
%		{2009}
%		{The Virgil Schrock Scholarship}
%		{Dept of Nuclear Engineering, UC Berkeley}
%	\entrymin
%		{2009}
%		{Member of Tau Beta Pi}
%		{National Engineering Honor Society}
%\end{entrylist}

\section{relevant experience}
\begin{entrylist}
	\entry
		{Jan `15 -- Present}
		{Associate Specialist}
		{Lawrence Berkeley National Laboratory/UC Berkeley}
		{
		Radiological Multi-sensor Analysis Platform (RadMAP)
		\begin{itemize}
			\item Upgraded gamma-ray detector acquisition hardware and software on board a mobile platform
			\item Utilized Python to develop a real-time analysis platform to process radiological (gamma-ray) and contextual (video and GPS) sensor data
			\item Produced fused data products for field operations and anomalous source detection scenarios
			\item Presented project progress and results at sponsor review meeting
		\end{itemize}
		DoseNet
		\begin{itemize}
			\item Project Co-Lead of a distributed radiation sensor network for educational purposes
			\item Managed the technical development of sensor hardware, the database back-end and the web visualization front-end
			\item Coordinated with local schools to mount sensors and build related educational modules
		\end{itemize}
		}
	\entry
		{Aug `12 -- Dec `14}
		{Graduate Student Researcher}
		{Lawrence Berkeley National Laboratory/UC Berkeley}
		{Radiological Multi-sensor Analysis Platform (RadMAP)
		\begin{itemize}
			\item Constructed detailed system models in MCNP and Geant4 for ray-tracing and Monte-Carlo gamma-ray simulations on high-performance computing clusters
			\item Conducted field experiments to validate simulation results
			\item Processed large data sets to produce visualizations for reports and communicated results
		\end{itemize}
		}
	\entry
		{Aug `11 -- Aug `12}
		{Junior Specialist}
		{Dept of Nuclear Engineering, UC Berkeley}
		{Advanced Concepts in Radiation Detection (NE204)
		\begin{itemize}
			\item Maintained experiments in a graduate radiation detection laboratory course
			\item Developed signal processing and gamma-ray imaging tools in MATLAB
%			\item Designed and machined experimental setup components
			\item Taught students methods to operate digital acquisition systems
		\end{itemize}
		}
\end{entrylist}

%\section{leadership}
%\begin{entrylist}
%	\entry
%		{2009 -- 2014}
%		{Engineering for Kids Day (E4K)}
%		{College of Engineering, UC Berkeley}
%		{
%		Coordinated logistics and safety for 300+ children during annual engineering outreach program.
%		}
%	\entry
%		{2012 -- 2013}
%		{Head Graduate Student Instructor}
%		{Dept of Nuclear Engineering, UC Berkeley}
%		{
%		Led three other instructors to setup, conduct, teach and assess undergraduate course experiments.
%		}
%	\entry
%		{2008 -- 2011}
%		{American Nuclear Society}
%		{Student Chapter, UC Berkeley}
%		{
%		Managed operations, organized outreach and liaised with national chapter as secretary, vp and president.%
%		}
%	\entry
%		{2009 -- 2010}
%		{Engineers Joint Council}
%		{College of Engineering, UC Berkeley}
%		{
%		Facilitated annual budget allocation of college-wide funding for engineering student groups.
%		}
%	\entry
%		{2009}
%		{Tau Beta Pi}
%		{Student Chapter, UC Berkeley}
%		{
%		Managed external relations with other student societies to foster collaborative relationships.
%		}
%\end{entrylist}

\section{relevant skills}
%\textbf{Programming:} Python, MATLAB, git, Bash, HDF5, SLURM, \LaTeX, CSS \& HTML\\
%\textbf{Techniques:} Scalable high performance computing, Monte-Carlo methods, uncertainty analysis, gamma-ray spectroscopy
\begin{DESCRIPTION}
	\item[Radiation Detection] Five years of hands-on experience with radiation detectors ranging from high-purity germanium gamma-ray spectroscopy to gamma/neutron pulse discrimination with liquid scintillators
	\item[Digital Acquisition Systems] Proficiency with modern digital acquisition hardware and software for gamma ray detectors
	\item[Scientific Workflows] Extensive experience with independent and cooperative data analysis using Python and Github as well as data storage techniques (HDF5), scientific visualizations (Matplotlib and Mayavi) and familiarity with machine learning techniques (Scikit-Learn)
	\item[Modeling and Simulation] Working knowledge of Monte-Carlo simulation techniques for gamma-ray physics using MCNP and Geant4 on high-performance parallel processing computing resources
	\item[Real-time Sensor Integration] Detailed understanding of processing and fusing multi-sensor data for analysis with detection algorithms and producing high quality data products
\end{DESCRIPTION}

%\section{relevant publications}

%\section{interests}
%
%\textbf{professional:} big data analysis, monte carlo simulations, machine learning, medical imaging technologies\\\textbf{personal:} photography, rock climbing, backpacking, cycling, open source coding

%----------------------------------------------------------------------------------------

\end{document}
